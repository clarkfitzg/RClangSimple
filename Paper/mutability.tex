\subsection{Determining unmodified inputs to a routine}\label{sec:MutabilityEg}

We mentioned in section~\ref{sec:Introduction} that it is can be
useful to determine if a routine modifies the memory pointed to by its
arguments.  Ideally, programmers would qualify parameters with a
\Ckeyword{const} declaration to indicate that it is not
modified. However, when this is not provided we do not know if it was
omitted or if the routine does actually change a parameter.
If we knew that it did not, we could avoid making a copy
of the inputs before calling it. 
In this example, we will build a simple-minded mechanism to try to
determine which parameters are modified within a routine.

%Similarly, if we could determine
%whether the routine allocated memory or expected to be allocated
%by the caller, we 

\noindent
\begin{figure}[ht]
\begin{minipage}[t]{0.45\linewidth}
\centering
\begin{CCode}
int
foo(int *x, int len)
{
    for(int i = 0; i < len; i++)
	x[i] = 2*x[i];
    return(len);
}

\end{CCode}
\caption{This routine modifies the memory pointed to by its
\Carg{x} argument.}
\label{fig:Mutability1}
\end{minipage}
\hspace{0.5cm}
\begin{minipage}[t]{0.45\linewidth}
\centering
\begin{CCode}
int
bar(int *x, int len)
{
    int total = 0;
    for(int i = 0; i < len; i++)
	total = 2*x[i];
    return(total);
}
\end{CCode}
\caption{Like \Cfunc{foo} on the left, this routine also 
expects a pointer to a \Ctype{int}, but it does not modify 
the contents of the memory pointed to by the argument.}
\label{fig:Mutability2}
\end{minipage}
\end{figure}

In this example, we show an alternative approach to traversing nodes
in the tree. We use a visitor but within the visitor explicitly
subset the children and their children with a single function call.

Consider the two routines in
figures~\ref{fig:Mutability1}~and~\ref{fig:Mutability2}.  Both take a
pointer to a collection of \Ckeyword{int} values.  The first changes
each element of the array. The second merely reads the elements of the
array. From the declaration, we cannot tell the difference.  We will
write a visitor function that traverses the code of the routine and
identifies expressions which modify one of the (pointer) parameters.
For simplicity, we will identify the names of the parameters of
interest, however, our function could compute the parameters as it
traverses the entire tree for the routine.  We'll start with a simple
visitor function, defined in a generator function to create the
closure. We can define this as 
\begin{RCode}
update =
function(cur, parent)  {
  k = cur$kind
  if(k == CXCursor_BinaryOperator && "=" %in% (toks <- getCursorTokens(cur)) ) 
      processLHS(cur[[1]])

  CXChildVisit_Recurse
}
\end{RCode}
%$

The basic strategy is that for each cursor, we determine if it is an
assignment operator. There is no simple function to do this in the
\libclang{} API, but we can combine different pieces of information to
get the answer. We check the kind of the cursor is
\Rvar{CXCursor_BinaryOperator} and then use \Rfunc{getCursorTokens} to
find the actual operator as a token.  If it is a \dquote{=}, we have
indentified our condition.  Using cursor tokens is a little less
precise than querying a specific cursor. The tokens may include
additional text before and after our cursor that may be misleading.

Once we have a binary operator, we are interested in its left-hand
side, i.e. the first child.  Rather than setting a state variable and
continuing to traverse the sub-cursors, we will explicitly process
this first child. We may have to traverse its children, but we will do
that with a separate visitor function -- \Rfunc{processLHS}.  This may
make the computations a little simpler.  So in our top-level visitor,
we access the first child of the current cursor with \Rexpr{cur[[1]]}
and  then call the function that will determine what is being modified.

Our next task is to define the \Rfunc{processLHS} function.  We could
define another visitor function, but instead we will manually traverse
the cursor tree.  We have various different situations to consider.
This left-hand side can be arbitrary \C{} code. Examples of the entire
expression include
\begin{CCode}
x[i] = 2*x[i];
total = 2*x[i];
*(x + i) = 2*x[i];  
y[ x[i] ] = 2*x[i];
*addr(y, x, i) = 2*x[i];
y[ f(x, i, y) ] = 2*x[i];
\end{CCode}
These illustrate assigning to an element, a variable,
a unary operator indexing \Cvar{x}, nested subsetting, 
calling a routine to get an address, and subsetting
with the index determined by a call to a routine.
The last two of these call other routines and we would have
to analyze their code to determine what is being modified.

We'll focus on the first four examples above.  We define a function
that can handle each of these different casesm.  \Rfunc{processLHS}
will be called with the cursor corresponding to the left-hand side of
the assignment.  We check to see if this is an array subscript
(\verb+[+), a variable reference (\verb+total+), a unary operator
(\verb+*+), another binary operator (\verb|x+i|), a call expression
(\verb+f(x, i, y)+ or \verb+addr(y,x, i)+) and so on.
The function is defined as
\begin{RCode}
processLHS = 
function(cur) {
  k = cur$kind

  id = if(k == CXCursor_ArraySubscriptExpr) { # x[i], y[x[i]], x[foo()]
         processLHS(cur[[2]])
         getName(cur[[1]])
       } else if(k == CXCursor_UnaryOperator){
         if(cur[[1]]$kind == CXCursor_ParenExpr)
           return(processLHS(cur[[1]][[1]]))
         else if(cur[[1]]$kind == CXCursor_DeclRefExpr)
           getName(cur[[1]])
         else if(cur[[1]]$kind == CXCursor_CallExpr)
             calls <<- c(calls, getName(cur[[1]]))
       } else if(k == CXCursor_FirstExpr || k == CXCursor_DeclRefExpr) {  
           # recursively from *( x + i) with x and again with i
         getName(cur)
       } else if(k == CXCursor_BinaryOperator) { # x + i arising in *(x + i)
         return(sapply(children(cur), processLHS))
       } else if(k == CXCursor_CallExpr)
             return(calls <<- c(calls, getName(cur)))
  
  if(!is.null(id) && id %in% paramNames)
      modifiedVars <<- c(modifiedVars, id)
} 
\end{RCode}
This function does ``visit'' the child cursors. Instead, it actively
extracts and processes them without returning control to the
\libclang{} iterator. This makes the logic a little easier to
understand.

Note that this function is defined within our generator function so
that it has access to the \Rvar{modifiedVars} variable and also so
that the \Rfunc{update} function can refer to it.

This visitor function identifies the basic assignments to 
parameters.  It identifies cases it cannot understand due to calls to
other routines to identify what is being assigned. 
It does not handle simple cases of aliasing such as
\begin{CCode}
void fun(int *x)
{
  int *y = x;
  y[0] = 10;
}
\end{CCode}
Here, we have a local variable that corresponds to \Cvar{x} and then
we modify its contents.  We can extend our function to identify these
cases.


