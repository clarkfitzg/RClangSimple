\section[Concepts of libclang]{Concepts of \libclang}\label{sec:libclangConcepts}

%Translation units,  AST and cursors, traversing the tree, types
Before we discuss the \R{} facilities for working with \libclang, it
is useful to understand the basic concepts of the library.  These
include translation units, abstract syntax trees and cursors, types
and tokens.  \libclang{} exposes other concepts, but those are not  the
important ones for our purposes.

Source code is arranged in files. A project may be made up of more
than one related files. %XXX grammar
For example, we may define related routines
for one task in one file and other routines in a separate file.
Header files are used for declarations and pre-processor definitions
that are shared by several files.  When we parse a file, the parser
essentially reads the source and substitutes the content of any header
files referenced by an \Ckeyword{\#include} directive directly into
the source.  The entity resulting from parsing the document is called
a \textit{translation unit} (TU).


\libclang{} is a parser and represents a translation unit as a parse
tree where the nodes are called \textit{cursors}.  It first breaks the
text into tokens during the lexical analysis step.  This breaks code
of the form \Cexpr{dnorm(n, 1, sd)} into the separate tokens
\dquote{dnorm}, \dquote{(}, \dquote{n}, \dquote{1}, \dquote{sd},
\dquote{)}.  The parsing stage then maps the tokens into language
concepts represented by different kinds of cursors.  In this example,
the concept is a call expression. Within the call entity is the
reference to the routine being called -- \Cfunc{dnorm} -- and then the
three arguments to the call.  The parenthesis and comma tokens
disappear in the parsing stage as we move from the individual tokens
to higher-level semantic meaning of the tokens.  However, a cursor
still has access to its original tokens and we can obtain these for a
given cursor. This turns out to be important as \libclang{} doesn't
differentiate between different types of binary operators, for
example. Instead, we find the operation (=, +, -, etc.)  from the
tokens and so the concepts of both cursor and token are important.
%XXX the last few sentences could be written better.  Too specific
% so make the point more gracefully.

\libclang{} represents the different parse elements 
%XXX elements of the parsed content
via a cursor (of
class \Rclass{CXCursor} in \R).  A cursor has a kind that identifies
its nature or what concept it represents. There are many different
kinds for a cursor (enumerated in the variable \Rvar{CXCursorKind} in
the \Rpkg{RCIndex} package) and they include FunctionDecl, VarDecl,
ParmDecl, CallExpr, IntegerLiteral, ClassDecl, MacroDefinition,
EnumDecl, StructDecl.  Hopefully the names indicate the purpose of the
cursor.

\begin{figure}
\begin{tikzpicture}
[
font=\tiny,
level distance=1.25cm,
level distance=1.25cm,
every node/.style={top color=white,
    bottom color=blue!25,
    rectangle,rounded corners,
    minimum height=8mm,
    draw=blue!75,
    very thick,
    drop shadow,
    align=center,
    text depth = 0pt},
 level/.style={sibling distance=6cm},
 level 4/.style={sibling distance=7cm},
 level 5/.style={sibling distance=7cm},
 level 6/.style={sibling distance=4cm},
 level 7/.style={sibling distance=3cm}
]
 \node {\Verb!double fun ( double x )  return ( sin ( x ) + log ( x + 1 ) ) ;!\\ FunctionDecl}
 child {
  node {\Verb!double x!\\ ParmDecl}
 }
 child {
  node {\Verb! return ( sin ( x ) + log ( x + 1 ) ) ;!\\ CompoundStmt}
  child {
   node {\Verb!return ( sin ( x ) + log ( x + 1 ) )!\\ ReturnStmt}
   child {
    node {\Verb!( sin ( x ) + log ( x + 1 ) )!\\ ParenExpr}
    child {
     node {\Verb!sin ( x ) + log ( x + 1 )!\\ BinaryOperator}
     child {
      node {\Verb!sin ( x )!\\ CallExpr}
      child {
       node {\Verb!sin!\\ FirstExpr}
       child {
        node {\Verb!sin!\\ DeclRefExpr}
       }
      }
      child {
       node {\Verb!x!\\ FirstExpr}
       child {
        node {\Verb!x!\\ DeclRefExpr}
       }
      }
     }
     child {
      node {\Verb!log ( x + 1 )!\\ CallExpr}
      child {
       node {\Verb!log!\\ FirstExpr}
       child {
        node {\Verb!log!\\ DeclRefExpr}
       }
      }
      child {
       node {\Verb!x + 1!\\ BinaryOperator}
       child {
        node {\Verb!x!\\ FirstExpr}
        child {
         node {\Verb!x!\\ DeclRefExpr}
        }
       }
       child {
        node {\Verb!1!\\ FirstExpr}
        child {
         node {\Verb!1!\\ IntegerLiteral}
        }
       }
      }
     }
    }
   }
  }
 }
;
\end{tikzpicture}
  
\cprotect\caption{This shows the structure of a cursor tree for the 
\C{} routine \Cfunc{sinLog} shown in figure~\ref{fig:sinLogRoutine}.
Each node in the tree shows the kind of the cursor
and the text of the expression with which it is associated.
The cursor kind \dquote{FirstExpr} essentially means the cursor is considered
opaque or hidden.
}
\label{fig:sinLogTree}  
\end{figure}

\begin{figure}
\centering
\begin{CCode}
              double
              sinLog(double x)
              {
                 return(sin(x) + log(x + 1));
              }
\end{CCode}  

\caption{A simple routine that takes a single \Ctype{double} value
and returns a new value. Figure~\ref{fig:sinLogTree} illustrates the
parse tree for this routine.}\label{fig:sinLogRoutine}
\end{figure}


Along with its kind, a cursor has child cursors.  These are the
components such as the reference to the routine and the parameters in
a call, the fields and methods in a \Cpp{} class or the left and right
hand side of an assignment expression. Each of these child cursors has
a kind and also, possibly, its own sub-cursors. As a result, we have a
tree, or hierarchical structure.  Figure~\ref{fig:sinLogTree}
illustrates this hierachy for a simple routine shown in
figure~\ref{fig:sinLogRoutine}, along with the kind of cursor and 
its expression.  The translation unit is the
container for the top-most or root cursor of the tree.  This root
cursor has the abstract kind \ClangKind{TranslationUnit}.  Its
children are the top-level elements of the source code file,
e.g., global variables, routines, class definitions, pre-processor
terms. Each of these has its own child cursors and so on.

This concept of a cursor tree is important when we examine how to
extract information using the lower-level facilities in
sections~\ref{sec:BuildingBlocks}~and~\ref{sec:Examples}.  We will traverse
the tree, either one node at a time or directly by querying child
cursors and their children.

In addition to cursors, \libclang{} manages type information.  In the
\C{} and \Cpp{} languages, every expression has a type, be it a
variable declaration, a call to a routine, a binary operator, and so on.  \libclang{}
associates a type with each cursor.  Importantly, it ensures that
there is a single type object describing each unique data type in the
translation unit, and across related translation units.  This allows
us to reason about and resolve types quite easily.

Like a cursor, a \libclang{} type has a kind such as \ClangTypeKind{Int},
\ClangTypeKind{Float}, \ClangTypeKind{Double},
\ClangTypeKind{Typedef}, \ClangTypeKind{Pointer},
\ClangTypeKind{Record} (for a \Ckeyword{struct}),
\ClangTypeKind{Enum}.  A type also has a name, a size (number of
bytes) and other characteristics we can query.


A translation unit contains all the information from the corresponding
source file.  A single TU is often sufficient for our purpose of
exploration and analysis. However, if the code in one translation unit
refers to routines or variables in an other source file, we have to
merge it with another translation unit to resolve those references.
\libclang{} uses an \textit{index} as the container for several
related translation units.  When we want to deal with the translation
units together, we use the same index to parse them and this then
allows \libclang{} and us to connect references to elements across
translation units.


\libclang{} provides other features such as diagnostics for both parsing
and compiler errors, code completion used in editors, serializing
translation units, precompiled header files and efficient indexing of
one or more translation units.  Some of these are exposed in the \Rpkg{RCIndex}
package. However, none are essential for understanding how to work
with \libclang{} and extracting useful information from source code.




