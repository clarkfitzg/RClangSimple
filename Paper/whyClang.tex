% This could come near the end.
%\section[Why libclang?]{Why \libclang?}\label{sec:Whylibclang}
\section{Other Research and Approaches}

I have explored how we can get information about \C/\Cpp{} code in
\R{} for many years and have used various different approaches.
\libclang{} appears to be the most robust, stable and flexible so far.
\clang{} and \libclang{} are very active projects and important
elements of the modern tool chain on different platforms. This means
they will continue to improve, evolve and be maintained.  The API
(Application Programming Interface) is intentionally stable.  As we
will see later, \libclang{} does the hard work in resolving types
uniquely and simplifying the complexities of native code.  \libclang{}
allows us to readily associate concepts in the parse tree with precise
locations in the text of the source code. \libclang{} doesn't allow us
to modify the parse tree and rewrite native code. However, there is a
quite different and more advanced API related to \libclang{} that does
and some of the concepts will carry over to that, should we need those
facilities in \R.

In the past, I have used the GNU compiler (\gcc)~\cite{bib:GCC} to
export information about the source code using its
\ShFlag{-fdump-translation-unit} command-line argument.  This outputs
many of the details of the source code in a reasonably low-level
format. I developed the \Rpkg{RGCCTranslationUnit}~\cite{bib:RGCCTU}
package to both parse the output from \gcc{} and also to collate the
information into higher-level descriptions of routines, data
structures, and generate binding code.  Unfortunately, the format of
\gcc's output is not well document, and the information it exports has
changed over releases.  Generally, this a valuable source of
information, but not as stable or as readily usable or developed as
\libclang.

GCC-XML is an extension to \gcc{} that, like the
\ShFlag{-fdump-translation-unit} option, can export information about
routines and data structures contained in \C/\Cpp{} source code.  As
the name suggest, it exports in an \XML{} format.  We can then read
this into \R{} with, for example, the \Rpkg{XML}~\cite{bib:RSXML}
package.  We can convert the \XML{} content into descriptions of the
routines and types.  Again, we have to implement this type registry.
Also, GCC-XML does not give us access to the bodies of the routines
and so we cannot address several of the applications we discussed in
section~\ref{sec:Introduction}.  Since GCC-XML does not come with
\gcc{} itself, one has to install it separately.  While this is not
very difficult, it is another hurdle.


I also explored SWIG (Simplified Wrapper and Interface
Generator)~\cite{bib:SWIG} as a means to programmatically generate
interfaces to native code. The idea was to leverage a widely used tool
that could generate code for different languages.  SWIG is written in
\perl{} and was somewhat complex. It is more desirable to write the
code generation mechanism in \R{} itself rather than another language.
This is because it is more familiar and also because more users can
understand the code and contribute and extend the development.  Also,
as the name suggests, the focus of SWIG is to generate bindings to
existing native code.  It does not allow us to examine the body of the
routines.


Many years ago, we adapted and embedded the little \C{} compiler
(lcc)~\cite{bib:lcc} in the \S{} language (both \R{} and \Splus).  This gave us many of the
features we obtain with \libclang, allowing us to access information
about routines, data structures, etc. in source code.  It does not
support \Cpp. Unfortunately, it did not readily deal with extensions
to the \C{} language used in the Linux header files. As a result,
other than extending the code base, we couldn't continue to use it for
our purposes.
 
Microsoft Windows provides facilities for accessing the information
about routines and data structures in native code via its type
libraries.  I developed the
\Rpkg{SWinTypeLibs}~\cite{bib:SWinTypeLibs} package to read this
information generally. This gives us access to much of the information
we need. Of course, it is specific to Microsoft Windows. It works on
previously compiled code, not the source code.  Accordingly, it gives
us no access to the bodies of the native code.


% SWIG
% gcc-xml
% Slcc
% Type libraries on Windows.

All of the approaches are reasonable and have their
advantages. However, \libclang{} is a significant improvement over all
of these other approaches for various different reasons. It provides
most of the infrastructure we need and removes the burden for
consumers to create this.  It is flexible, but not excessively complex.
It is tied to serious, ongoing projects and so continues to evolve. It
allows us to work in the language in which we want to program,
i.e. \R.  It provides access to all aspects of the native code, not
just the declarations. It processes both \C{} and \Cpp.

The approach we described in this paper does not work when we don't
have the source code available, but only compiled binary libraries. We
will have header files available that provide the declarations of the
routines and the data types of interest in the library.  We can use
these for many of the purposes. We cannot of course examine the bodies
of the routines.

