\subsection[Checking for garbage collection errors in native code for R]{Checking for garbage collection errors in native code for \R{}}\label{sec:ProtectEG}
In this example, we will write code that can examine a \C{} routine
written to be called via \R's \Rfunc{.Call} interface and try to
identify if there are possible errors related to ensuring \R{} objects
are not garbage collected prematurely.  The \R{} API uses the macros
\Cfunc{PROTECT} and \Cfunc{UNPROTECT} to mark an object as being in
use and stop if from being garbage collected and then to unmark a number of
protected objects.  For example, the \C{} code in
figure~\ref{fig:ProtectCorrect} creates two new \R{} objects and
protects them both, performs some computations that populate the
objects and then unprotects both with a call \Cexpr{UNPROTECT(2)}.  In
contrast, the code in figure~\ref{fig:ProtectIncorrect} does not
protect the \R{} object it creates.  It is quite possible that after
allocating \Cvar{ans}, \R{} will release that object when it allocates
\Cvar{names} in the next expression. At that point, the memory is
corrupted and errors and crashes are likely.
In other cases, we might protect the \R{} objects we create,
but fail to unprotect them, or at least some of them.

\noindent
\begin{figure}[ht]
\begin{minipage}[t]{0.45\linewidth}
\centering
\begin{CCode}
SEXP
R_foo_correct(SEXP r_x)
{
  SEXP ans, names;
  int n = Rf_length(r_x);
  double *x = REAL(x);
  PROTECT(ans = NEW_NUMERIC(n));
  PROTECT(names = NEW_CHARACTER(n));
  for(int i = 0; i < n; i++) {
    char *str;
    REAL(ans)[i] = foo(x[i], &str);
    SET_STRING_ELT(names, i, 
                     mkChar(str));
  }
  SET_NAMES(ans, names);
  UNPROTECT(2);
  return(ans);
}
\end{CCode}
\caption{This is a simple and correct \C{} routine
that protects and unprotects the \R{} objects it creates.}
\label{fig:ProtectCorrect}
\end{minipage}
\hspace{0.5cm}
\begin{minipage}[t]{0.45\linewidth}
\centering
\begin{CCode}
SEXP
R_foo_incorrect(SEXP r_x)
{
  SEXP ans, names;
  int n = Rf_length(r_x);
  double *x = REAL(x);
  ans = NEW_NUMERIC(n);
  names = NEW_CHARACTER(n);
  for(int i = 0; i < n; i++) {
    char *str;
    REAL(ans)[i] = foo(x[i], &str);
    SET_STRING_ELT(names, i, 
                       mkChar(str));
  }
  SET_NAMES(ans, names);
  return(ans);
}

\end{CCode}
\caption{This version does not protect the \R{} objects and, hence,
  also  doesn't unprotect them.}
\label{fig:ProtectIncorrect}
\end{minipage}
\end{figure}


%XXX Clean this up wrt. to macro expansions. Ignore them  but just
%mention it is possible.
We will develop an \R{} function that will attempt to check for common
problems related to garbage collection.  Our function takes the
\libclang{} cursor for a routine and will then traverse the cursors
throughout that tree. It will find calls to known routines that
allocate \R{} objects (e.g. \Cfunc{Rf_allocVector},
\Cfunc{NEW_INTEGER}, \Cfunc{NEW_NUMERIC}) and also calls to
\Cfunc{Rf_protect} and \Cfunc{PROTECT}, and \Cfunc{Rf_unprotect} and
\Cfunc{UNPROTECT}.  A simple test for correct code is to count the
number of allocations, the number of calls to
\Cfunc{Rf_protect}/\Cfunc{PROTECT} and attempt to determine the value
passed to calls to \Cfunc{Rf_unprotect}/\Cfunc{UNPROTECT}.  If these
don't match, we can indicate a potential error.  This is a simple
heuristic approach, but it identifies many common cases.

\begin{comment}
What about \Cfunc{NEW_NUMERIC}, \Cfunc{PROTECT} and \Cfunc{UNPROTECT}?
In some cases, these are acutally pre-processor macros and in others
they are routines. They expand to
\Cexpr{Rf_allocVector(REALSXP, n)} and calls to \Cfunc{Rf_protect} and
\Cfunc{Rf_unprotect}.  We can find these expansions and substitutions
in the translation unit. Alternatively, we can use our knowledge of
how the \R{} API works.
\end{comment}

Since we will traverse over the sub-cursors of the routine, we need a
visitor function. We'll use a closure, but again, we could use a
reference class.  We need variables in which we can record the number
of calls to each of categories of routines that allocate, protect and
unprotect objects.  We'll call these variables \Rvar{numAllocs},
\Rvar{numProtectCalls} and \Rvar{numUnprotectCalls}, respectively.
For the first two categories of routines, we just increment the
current value in the corresponding \R{} variable.  For
\Cfunc{Rf_unprotect}, we'll collect the argument for each call to
\Cfunc{Rf_unprotect}. This might be a literal value, e.g. $2$ in our
example, or a variable or a general expression which makes our taks
more complex.

We have now identified the basic strategy for our visitor function.
Next, we need to specify the details for the different types of
cursors we encounter. A call to any of our routines in our categories
of interest will be identified by a cursor with a kind
\Rvar{CXCursor_CallExpr}.  We can get the name of the routine being
called using \Rfunc{getName} applied to this cursor. (There are other
ways also, but this is the simplest.)  Based on this name, we update
the relevant counter or ignore the call. 

If the call in the cursor is to \Cfunc{Rf_unprotect} or
\Cfunc{UNPROTECT}, we have to arrange to get the call's first and only
argument.  We can use either of two approaches for this.  We can set a
flag in our closure to indicate we are processing the sub-cursors of a
\Cfunc{Rf_unprotect} call and then subsequent calls to our visitor
function can check this and interpret the sub-cursors
appropriately. Alternatively, we can explicitly traverse the children
of the current call cursor within our visitor function to determine
the argument.  We'll use the former approach as it illustrates setting
and unsetting state across calls. We'll illustrate the second approach
later in this example.

We define our generator  function as
\begin{RCode}
R_AllocRoutineNames = c("Rf_allocVector", "NEW_NUMERIC", "NEW_INTEGER",
                        "NEW_LOGICAL", "NEW_CHARACTER", "NEW_LIST")

genProtectAnalyzer = function() 
{
  numAllocs = 0  
  numProtectCalls = 0
  numUnprotectCalls = numeric()

  inUnprotect = FALSE  
  allocCounterVarName = ""    
  unProtectParent = NULL

  update = function(cur, parent)    {
    if(inUnprotect && identical(unProtectParent, cur) ) {
       unProtectParent <<- NULL
       inUnprotect <<- FALSE
    }

   k = cur$kind    # get the kind of this cursor
   if(k == CXCursor_CallExpr) {
      fn = getName(cur) # name of the routine being called
      if(fn == "PROTECT" || fn == "Rf_protect")
        numProtectCalls <<- numProtectCalls + 1
      else if(fn == "UNPROTECT" || fn == "Rf_unprotect") {
        numUnprotectCalls <<- numUnprotectCalls
        unProtectParent <<- parent
        inUnprotect <<- TRUE
      } else if(fn %in% R_AllocRoutineNames)
          numAllocs <<- numAllocs + 1
    } else if(inUnprotect) {
       if(k == CXCursor_IntegerLiteral) {
          val = getCursorTokens(cur)["Literal"]
          numUnprotectCalls <<-  c(numUnprotectCalls, as.integer(val))
       } else if(k == CXCursor_FirstExpr) {
           allocCounterVarName <<- getName(cur)
       }
    }
    
    CXChildVisit_Recurse  # visit this cursor's children
  }

  list(update = update,
       info = function() {
                 list(numProtectCalls = numProtectCalls,
                      numUnprotectCalls = numUnprotectCalls,
                      inUnprotect = inUnprotect,
                      numAllocs = numAllocs)})
}
\end{RCode}
%$
For the present, ignore the first expression in our visitor function,
the \Rfunc{if} expression.  We'll discuss this below.  Our code
examines the kind of the cursor and if it is \Rvar{CXType_CallExpr} it
does one thing and otherise does another.  For a call, we check which
category of interest it is in, or not and update the relevant
variables.  For an unprotect call, we set the variable
\Rvar{inUnprotect} so that subsequent calls to this function will
recognize this and extract the information from the argument to the
unprotect call.

If the cursor is not a call expression, then we check to see if
\Rvar{inUnprotect} is \Rtrue.  If it is, then we attempt to interpret
the current cursor as (part of) the argument to the unprotect routine.
Our goal is to get the value of that argument.  Our simplified version
checks to see if the current cursor is a
\Rvar{CXCursor_IntegerLiteral} type or a \Rvar{CXCursor_FirstExpr}.
The former indicates that we have a literal integer value.
Unfortunately, there is no function in the \libclang{} API for us to
access the associated value directly. Instead, we use the
\Rfunc{getCursorTokens} function. This gives us the text around the
cursor, broken into tokens or atomic elements that make sense at the
\C/\Cpp-level.  The elements in the resulting character vector are
named with their token types and so we can extract the
value named \dquote{Literal}.  If the cursor is so-called
\Rvar{CXCursor_FirstExpr}, we get its name as the value of a
variable. In reality, we would process its sub-cursors.

The logic is quite simple -- deal with a call expression or, if
\Rvar{inUnprotect} is \Rtrue{}, deal with other kinds of cursors.
This raises the question as to when \Rvar{inUnprotect} is ever set
back to \Rfalse.  This brings us back to the initial \Rfunc{if}
expression in the function:
\begin{RCode}
if(inUnprotect && identical(unProtectParent, cur) ) {
   unProtectParent <<- NULL
   inUnprotect <<- FALSE
}
\end{RCode}
The purpose of this is to ensure that we don't continue to collect
information from other cursors after we have processed any call to
\Cfunc{UNPROTECT}.  If we didn't have this, the variable
\Rvar{inUnprotect} would remain \Rtrue{} even after we have process
the entire call expression to the unprotect routine.  As a result, if
there are \C{} expressions in the body of the routine after the call
to \Cfunc{UNPROTECT} and they contain literal values or
\Rvar{CXCursor_FirstExpr} cursors, we will continue to accumulate
information from these as if they related to the \Cfunc{UNPROTECT}
call. Accordingly, we need to determine when we have finished
examining the sub-cursors of the \Cfunc{UNPROTECT} call.  To do this,
we record the parent cursor of the \Cfunc{UNPROTECT} call in
the variable \Rvar{unProtectParent}. Each time our visitor is called, we can
compare the current cursor to this cursor and if they are the same, we
are back to a sibling of the \Cfunc{UNPROTECT} call.

Setting and unsetting a state variable across calls to a visitor
function can be complex and often requires clear thinking.
We could have used the other approach which was to detect
the call to \Cfunc{UNPROTECT} and immediately determine its arguments
using \Rfunc{children}. For example, we could have added the code
\begin{RCode}
arg = children(cur)[[2]]
if(arg$kind == CXCursor_IntegerLiteral)
   as.integer(getCursorTokens(arg)["Literal"])
\end{RCode}
%$
This makes the visitor function a great deal simpler and also slightly
faster as we can avoid traversing the sub-cursors again.  However,
firstly, this doesn't take into account that the information may not
be as readily accessible via the immediate children but may be in
their descendants.  We could write a separate function to visit these
and call it from our visitor.  Secondly, sometimes we have to use
state, in particular when the information we need to extract is not in
sub-cursors but in silbing cursors and their descendants.  This is not
uncommon.


With our visitor function defined, we are now ready to use on
some routines. 
Before we can check a routine, we have to read it into \R{}.
We  do this with \Rfunc{getRoutines} via
\begin{RCode}
f = system.file("exampleCode", "protectIncorrect.c", package = "RCIndex")
r = getRoutines(f, f, includes = sprintf("%s/include", R.home()))
\end{RCode}
Note that we specified the include directories so that
the parser could find \file{Rinternals.h} and the other 
\R{} header files.

We create our visitor function by calling the generator function
\Rfunc{genProtectAnalyzer} and then we pass the visitor element to
\Rfunc{visitCursor}, along with the routine we want to check:
\begin{RCode}
v = genProtectAnalyzer()
visitCursor(r$R_foo_incorrect, v$update)
v$info()
\end{RCode}
The output is 
\begin{ROutput}
$numProtectCalls
[1] 0

$numUnprotectCalls
numeric(0)

$numAllocs
[1] 2
\end{ROutput}
We can see there are two allocations and no calls to protect or
unprotect \R{} objects.  
In constrast, when we run this on the correct version, 
each of the three counts has a value of $2$.

\begin{comment}
f = system.file("exampleCode", "protectCorrect.c", package = "RCIndex")
r = getRoutines(f, f, includes = sprintf("%s/include", R.home()))
v = genProtectAnalyzer()
visitCursor(r$R_foo, v$update)
v$info()
\end{comment}
%$



As a final remark about this example, we could make this more general
and sophisticated.  A reasonably common idiom is to use a variable to
count the number of calls to \Cfunc{PROTECT} we make, incrementing it
each time. Then we pass this to \Cfunc{UNPROTECT}, e.g.
\begin{CCode}
 int n = 0;
 PROTECT(a = Rf_allocVector(...)); n++;
 PROTECT(b = Rf_allocVector(...)); n++;
 PROTECT(c = Rf_allocVector(...)); n++;
 PROTECT(d = Rf_allocVector(...)); n++;

 UNPROTECT(n);
\end{CCode}
It is easy to omit incrementing the counter in one or more cases. We
can try to trace this by identifying the variable in the call to
\Cfunc{UNPROTECT} and then finding out where it is incremented and try
to find cases where it is not.  In our code above, this should be the
adjacent sibling of the call to \Cfunc{PROTECT}. This is an example of
requiring state across calls.  We should note that if we find the
\Cfunc{UNPROTECT} call uses a counter variable, we can make a second pass over the
cursor tree to find out where it is used.
% Show example code


