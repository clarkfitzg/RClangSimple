\subsection{Writing visitor functions}
%Closures, reference classes
%Reference class and inheritance.


%XXX Correct this to talk about traversing any cursor, not just the TU.
The primary way to extract information from a translation unit is to
use the \Rfunc{visitTU} function.  This takes the name of a source
file or an already parsed translation unit.  The second argument is a
``visitor'' function.  This can take various forms but
essentially correspnds to an \R{} function (or the address of a native
symbol).  \libclang{} iterates over some or all of the cursors in the
translation unit and calls the \R{} function for each each of these
cursors, passing it both the current cursor and its parent cursor, for
context.  The function can perform any computations it desires and
typically extracts any information it wants from each cursor, or the
cursor itself.
%XXXX
% Move this to after the closures and explain that we can always
% Recurse, but it can be a waste of time.
It then returns a
value to indicate to \libclang{} how it should continue to traverse
the cursor tree.  We can instruct \libclang{} to stop, to traverse
only the siblings of the current cursor, or to recursively process the
child cursors.  We can specify a different return value for each call
and so change how \libclang{} traverses the tree dynamically.  For
example, we might want to process all of the nodes in a particular
routine but no other routines. So we would return
\Rvar{CXChildVisit_Recurse} when we first encounter the \Rvar{CXCursor_FuncDecl}%??? 
with the name of interest.  When we encounter the start of an other routine, we return
\Rvar{CXCursor_Continue} or we can simply terminate the traversal with
\Rvar{CXCursor_Break}.

The visitor function typically stores information that persists across
the calls to it made by \libclang.  We then access these objects when
\libclang{} has completed traversing the tree.  While we may be
inclined to use global variables, e.g. in \R{}'s global environment or
interactive ``work space'', this is a bad idea.  Instead, we want to
use variables that are local to this particular visitor function.  We
can do this in two basic ways -- reference classes or closures, also
known as lexical scoping.  In fact, these two are very similar.

We'll focus on a very simple example which merely
iterates over the sub-cursors and finds the name and location of each
routine in the translation unit.  We are looking for cursors at the
top-level of the translation unit which have a cursor kind
\Rvar{CXCursors_FunctionDecl}.  We ignore any other cursors.

We will store the name of each routine and also its line number.
We have to put these into a non-local variable within a call to our
visitor  function and then be able to retrieve them. This is a closure.
Creating a closure is quite simple but requires understanding how \R{}
finds variables.  One approach to creating a closure is to define a
function that returns one or more functions.  We often call this a
generator function.
We can define our generator function something like
\begin{RCode}
genRoutineLocations = 
function() 
{
   locations = integer()

    visitor = function(cur, parent) {
       if(cur$kind == CXCursor_FunctionDecl) 
         locations[[ getName(cur) ]] <<- getLocation(cur)$location["line"]

        CXChildVisit_Continue
    }

    list(update = visitor, locations = function() locations)
}
\end{RCode}
There are several things to note.  Firstly, we have defined the
\Rfunc{genRoutineLocations} function. That is not what we will use to
traverse the tree. Instead, we call it to get a new pair of functions
returned in a list.  We will pass the \Rel{visitor} element of that
list to \Rfunc{visitCursor}.  Secondly, note the use of \verb+<<-+
when we assign the line number to the element of our \Rvar{locations}
vector. This is a non-local assignment an the updated value will
persist across calls to this function. Thirdly, note that the
\Rvar{locations} vector is defined in the body of
\Rvar{genRoutineLocations}, not within our visitor function. The
visitor function merely modifies it.  Fourthly, our second function in
the returned list (\Rel{locations}) is an anonymous function also
defined during a call to \Rfunc{genRoutineLocations} and it returns
the current value of the \Rvar{locations} variable when it is called.
This allows us to retrieve the results after they were added in the
different calls to our visitor function.  Lexical scoping and closures
are described in various texts about \R{} and \cite{bib:LexicalScoping}
is the definitive reference.

We can now use our generator function and obtain the line numbers of
the different routines. We do so with
\begin{RCode}
f = system.file("exampleCode", "fib.c", package = "RCIndex")
col = genRoutineLocations()
visitCursor(f, col$visitor)
col$locations()
\end{RCode}
Our call to \Rfunc{genRoutineLocations} returns the list of two
functions.  We pass the \Rel{visitor} element to \Rfunc{visitCursor},
along with the name of the source code file.  Then we retrieve the
updated contents with the second function \Rel{locations}.


As an aside, note that in this particular case, we don't need to
recursively descend through the cursors as the routines will all be
immediate children of the translation unit.  Therefore, our visitor
function returns \Rvar{CXChildVisit_Continue} to move to the next
sibling and not down across its children.  This saves times processing
irrelevant cursors.

Closures are very powerful but confuse some \R{} programmers.
Reference classes may be more familiar, especially to those familiar
with classes in \Cpp, \Java{} or \Python.  The idea is that we define
a class that has methods that share variables.  Our methods will be
our \Rel{visitor} and \Rel{locations} accessor functions in the generator
function above, and the fields or variables will be the
\Rvar{locations} numeric vector.  We can use the same code, but
aggregate it in a different way to define a reference class as:
\begin{RCode}
RoutineLocationVisitor =
setRefClass("RoutineLocationVisitor",
    fields = list(locations = "numeric"),
    methods = list(getLocations = function() 
                                     locations,
                   visitor = 
                       function(cur, parent) {
                         if(cur$kind == CXCursor_FunctionDecl) 
                            locations[[ getName(cur) ]] <<- 
                                   getLocation(cur)$location["line"]
                               
                         CXChildVisit_Continue
                       }))
\end{RCode}
Here, we explicitly identify the shared variable and the methods via
the \Rarg{fields} and \Rarg{methods} parameters.  The only real
difference between this and our generator function is that we have
to call our accessor function to retrieve the \Rvar{locations} value
\Rfunc{getLocations} rather than \Rfunc{locations}.  This is merely to
avoid have a field and a method with the same name.

We can now use this reference class in almost exactly the same way we
used our  generator function:
\begin{RCode}
col = RoutineLocationVisitor()
visitCursor(f, col$visitor)
col$getLocations()
\end{RCode}
The \Rfunc{RoutineLocationVisitor} function returned by
\Rfunc{setRefClass} creates a new instance of this reference class.
We pass the \Rfunc{visitor} method to \Rfunc{visitCursor} and then
call the function \Rfunc{getLocations} to obtain the results.



You can use any reference class you want and then pass the visitor or
update method to \Rfunc{visitTU}.  However, it can be convenient to
define your reference class to extend the \Rclass{RefCursorVisitor} class in
the \Rpkg{RCIndex} package. This allows you to pass the entire
reference object to \Rfunc{visitTU} and to get the results back
directly. This is merely syntactic sugar to simplify the
programming. It changes the code above to create the reference class,
call \Rfunc{visitCursor} and retrieve the results to the more succinct
\begin{RCode}
visitCursor(cur, RoutineLocationVisitor)
\end{RCode}
Essentially, extending the \Rclass{RefCursorVisitor} class allows
\Rfunc{visitCursor} to identify the visitor and the result functions.
The only change we need to make when defining our reference class
is to add the \Rarg{contains} argument in the call to
\Rfunc{setRefClass}, i.e.
\begin{RCode}
setRefClass("RoutineLocationVisitor",
            contains = "RefCursorVisitor",
            fields = list(locations = "numeric"),
            methods = list(getLocations = function() locations,
                       .....))
\end{RCode}



Similar to the \Rclass{RefCursorVisitor} reference class, we can use
an S4 class -- \Rclass{S4CursorVisitor} -- to combine the visitor and
result function and pass the two together to \Rfunc{visitCursor}.  We
typically create our two functions as before via a generator function,
and then combine the two functions into a formal object with, for
example,
\begin{RCode}
col = genCollector()
visitCursor( cur, new("S4CursorVisitor",  update = col$update, result = col$vars))
\end{RCode}
Again, the purpose of this \Rclass{S4CursorVisitor} class is merely to
allow \Rfunc{visitCursor} to identify the two functions -- the visitor
function and the function to access the results.

