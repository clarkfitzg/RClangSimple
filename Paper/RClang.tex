\documentclass[article]{jss}
\usepackage{comment}
%%%%%%%%%%%%%%%%%%%%%%%%%%%%%%
%% declarations for jss.cls %%%%%%%%%%%%%%%%%%%%%%%%%%%%%%%%%%%%%%%%%%
%%%%%%%%%%%%%%%%%%%%%%%%%%%%%%

%% almost as usual
\author{Duncan Temple Lang\\University of California at Davis}
\title{RCIndex: Accessing C/C++ Code in R with libclang}

%% for pretty printing and a nice hypersummary also set:
\Plainauthor{Duncan Temple Lang}
\Plaintitle{RCIndex: Accessing C/C++ Code in R with libclang}
\Shorttitle{RCIndex}

%% an abstract and keywords
\Abstract{ 
  The ability to get information about \C/\Cpp{} code routines
  and data structures can allow us to do many things in an intrepreted
  language such as \R.  We use \libclang{}, a flexible, embeddable
  library, to develop \R{} functionality to obtain information about
  native code.  We describe the \Rpkg{RCIndex} package which provides
  high-level functionality to access this information and also
  lower-level approaches to query and manipulate other aspects of
  native code.  We describe how to use the package and scenarios in
  which it is useful.
} 
\Keywords{\R, \Rpkg{RCIndex} package}
\Plainkeywords{R, RCIndex} %% without formatting
%% at least one keyword must be supplied

%% publication information
%% NOTE: Typically, this can be left commented and will be filled out by the technical editor
%% \Volume{13}
%% \Issue{9}
%% \Month{September}
%% \Year{2004}
%% \Submitdate{2004-09-29}
%% \Acceptdate{2004-09-29}

%% The address of (at least) one author should be given
%% in the following format:
\Address{
  Duncan Temple Lang\\
  4210 Math Sciences Building, \\
  University of California at Davis \\
  One Shields Avenue\\
  Davis, CA 95616\\
  E-mail: \email{duncan@r-project.org}\\
  URL: \url{http://www.omegahat.org}
}

\usepackage[T1]{fontenc}
\catcode`\_=12

\def\C{\proglang{C}}
\def\perl{\proglang{PERL}}
\def\Cpp{\proglang{C$++$}}
\def\Java{\proglang{Java}}
\def\Python{\proglang{Python}}
\def\R{\proglang{R}}
\def\llvm{\proglang{LLVM}}
\def\Rpkg#1{\pkg{#1}}
\def\Rfunc#1{\textsl{#1()}}
\def\Rop#1{\texttt{#1}}
\def\Rvar#1{\textsl{#1}}
\def\Rel#1{\textbf{#1}}
\def\Cfunc#1{\textit{#1()}}
\def\Cvar#1{\textit{#1}}
\def\file#1{\textbf{#1}}
\def\Ctype#1{\texttt{#1}}
\def\Rclass#1{\textit{#1}}
\def\Rslot#1{\textbf{#1}}
\def\Rarg#1{\textbf{#1}}
\def\Roption#1{\dquote{\textsl{#1}}}
\def\Cppkeyword#1{\textbf{#1}}
\def\Cexpr#1{\textsl{#1}}

\def\Rtrue{TRUE}
\def\Rfalse{FALSE}

\def\libclang{\textbf{libclang}}
\def\clang{\textbf{clang}}
\def\gcc{\textit{GCC}}
\def\XML{\textit{XML}}

\def\dquote#1{``#1''}

\def\ShFlag#1{\textit{#1}}

\DefineVerbatimEnvironment{RCode}{Verbatim}{fontshape=sl}
\DefineVerbatimEnvironment{CCode}{Verbatim}{fontshape=tt}
\DefineVerbatimEnvironment{ShCode}{Verbatim}{fontshape=it}
\DefineVerbatimEnvironment{ROutput}{Verbatim}{fontshape=bf}

\begin{document}

\section{Motivation}\label{sec:Introduction}

We typically think of \C{} and \Cpp{} code as something we write,
compile and call. It is rarely the input to anything but the compiler.
However, such code is a source of potentially useful information that
we can use in statistical and scientific computing.  In order to
leverage this information, we need to be able to access it in a
structured manner in a form that we can compute on and in the
programming environment in which we want to use it.
\libclang{}~\cite{bib:libclang,bib:libclangSlides} has
emerged as a powerful, industrial-strength library that provides
valuable functionality for working with \C/\Cpp{} source code and that
we can embed in \R (and many other languages).

There are several applications of being able to programmatically
understand \C/\Cpp{} code.

\textbf{Registering \R-callable routines}: When we write \C/\Cpp code
to use in an \R{} package, it is useful to explicitly register the
routines that we can call via any of the
\Rfunc{.C}/\Rfunc{.Call}/\Rfunc{.External} interfaces.  When the
routines are registered, \R{} can help to identify potential errors in
calling them.  \R{} can detect an incorrect number of arguments for a
routine, or that the types are incorrect, e.g. an integer vector when
a numeric vector is expected. Registration also allows us to resolve
the symbols just once rather than each time we call a routine, and it
also allows us to use different symbols to refer to the routines.  It
is convenient to be able to run \R{} code to identify the \R-callable
routines and to generate the registration information
programmatically.  As we change these routines, we can update the
registration information with little effort and ensure the information
is synchronized.


\textbf{Generating bindings to native libraries}: Numerous \R{}
packages provide interfaces to existing \C/\Cpp{} libraries.  This
typically involves manually creating an \R-callable wrapper routine
for each routine of interest in the third-party library and also a
corresponding \R{} function that invokes the wrapper routine, having
coerced the \R{} arguments to the appropriate form.  This is often
quite straightforward, but both time-consuming and error-prone.  This
makes for unnecessarily lengthy write-debug-test cycles.  Instead,
we'd like to be able to programmatically read the information about
the third-party routines and data types and generate the wrapper
routines and \R{} functions without human intervention, or at least
minimal, generic intervention.  If we could generate these
``bindings'' programmatically, the \R{} programmer can spend time
creating higher-level functionality using these primitives.

\textbf{Dynamic calls to native routines}: The
\Rpkg{rdyncall}~\cite{bib:rdyncall} and \Rpkg{Rffi}~\cite{bib:Rffi}
\R{} packages avoid having to explicitly create the wrapper routines
and \R{} functions to interface to existing \C{} routines. Instead,
they both provide a dynamic mechanism to call arbitrary native
routines.  However, both approaches require a description of the
target routines.  Again, we want to obtain this information
programmatically and then we can easily generate these descriptions
and remove humans from the process.

\textbf{Understanding third-party libraries interactively}: When we
interface to third-party libraries, we typically read documentation to
identify and understand the important routines and data structures.
In some situations, it can be convenient to find this information
interfactively within an environment such as \R.  Rather than reading
static document, we can query the code for information such as how
often a particular data type is returned by a routine or passed as an
argument? or what idioms does the library use?  We can use \R's
graphics capabilities to visualize the code and which routines call
which other routines.

\textbf{In-line documentation as comments}: Often third-party native
libraries contain document in comments adjacent to the corresponding
routines and data structures.  It is convenient to be able to easily
access this documentation and potentially reuse it as \R{}
documentation for functions that interface to the routines.

\textbf{Compiling \R{} code}:  Recently, we have been developing \R{} facilities for compiling \R{}
code to native instructions to by-pass the \R{} interpreter.  This
allows us to rewrite and translate \R{} code so that it is essentially
native code and can call other existing native routines, for example,
in the \R{} engine or standard \C{} libraries.  This results in
significant speedup.  However, to make this work, we need to know the
signature -- parameter types and return type -- of these native
routines.  Again, if we can find these programmatically, we greatly
simplify and improve the entire process of generating the code.

\textbf{Determining memory management and mutability of inputs and
  outputs}: When we call existing native routines, we often want need
to know whether we are in charge of the memory for the inputs or
whether the called routine is responsible.  Often, programmers omit
important information about whether a parameter is modified within a
routine or if it is constant.  This is important information as it
allows us to differentiate between an array of values passed as a
pointer, or a scalar whose contents are changed.  We would like to be
able to analyze the body of the routine to be able to determine if and
how it modifies its parameters so that we can avoid making copies of
data, if possible.


\textbf{Software as Data}: While we may not think of code as data,
analyzing software is an important field.  Software defines several
networks relatd to i) which routines call which other routines, ii)
the hierarchical structure of \Cpp{} classes, iii) which files
\texttt{\#include} other files, and so on.  We can find which
global/non-local variables are used in which routines to help identify
problems with parallelization and potential refactorzation of the
code.  We can combine this data with version control history to better
understand software projects.


\textbf{Detecting errors in native code for \R}: \R's package
mechanism provides a powerful set of tests and checks to potential
errors in \R{} code.  These are very useful and can identify errors
such as misspelled variable or function names before the code is run.
It would be valuable to perform similar tests on \C{} code in \R{}
packages.  We might identify common issues such as not protecting
allocated \R{} objects from garbage collection.  We could do this by
analyzing uses of \Cfunc{PROTECT} and \Cfunc{UNPROTECT} calls and
ensuring there is an appropriate correspondance.  These are
\R-specific checks and will not be done by other code-analyzing
software.


We can also find ``dead'' routines that are never called by other
routines and so help to reduce the code.

We hope these applications motivate the utility of being able to
navigate native code in \R and indicate that there are many more.  In
this paper, we describe how to use the \R{} interface to \libclang.
We start in section~\ref{sec:Whylibclang} by comparing \libclang{}
with other approaches we and others have pursued.  We then describe
the fundamental concepts of \libclang{} in
section~\ref{sec:libclangConcepts}.  We follow this in
section~\ref{sec:RCIndex} by discussing the \R{} interface

%documentation in comments - see RCUDA.

\section[Why libclang?]{Why \libclang?}\label{sec:Whylibclang}

I have explored how we can get information about \C/\Cpp{} code in
\R{} for many years and have used various different approaches.
\libclang{} appears to be the most robust, stable and flexible so far.
\clang{} and \libclang{} are very active projects and important pieces
of software. This means they will continue to improve, evolve and be
maintained.  The API (Application Programming Interface) is
intentionally stable.  As we will see later, \libclang{} does the hard
work in resolving types uniquely and simplifying the complexities of
native code.  \libclang{} allows us to readily associate concepts in
the parse tree with precise locations in the text of the source
code. \libclang{} doesn't allow us to modify the parse tree and
rewrite native code. However, there is a quite different and more
advanced API related to \libclang{} that does and some of the concepts
will carry over to that, should we need those facilities in \R.

In the past, I have used the GNU compiler (\gcc)~\cite{bib:GCC} to
export information about the source code using its
\ShFlag{-fdump-translation-unit} command-line argument.  This outputs
many of the details of the source code in a reasonably low-level
format. I developed the \Rpkg{RGCCTranslationUnit}~\cite{bib:RGCCTU}
package to both parse the output from \gcc{} and also to collate the
information into higher-level descriptions of routines, data
structures, and so on.  Unfortunately, the format of \gcc's output is
not well document, and the information it exports has changed over
releases.  Generally, this a valuable source of information, but not
as stable or as readily usable or developed as \libclang.

GCC-XML is an extension to \gcc{} that, like the
\ShFlag{-fdump-translation-unit} option, can export information about
routines and data structures contained in \C/\Cpp{} source code.  As
the name suggets, it exports in an \XML{} format.  We can then read
this into \R{} with, for example, the \Rpkg{XML}~\cite{bib:RSXML}
package.  We can convert the \XML{} content into descriptions of the
routines and types.  Again, we have to implement this type registry.
Also, GCC-XML does not give us access to the bodies of the routines
and so we cannot address several of the applications we discussed in
section~\ref{sec:Introduction}.


I also explored SWIG (Simplified Wrapper and Interface
Generator)~\cite{bib:SWIG} as a means to programmatically generate
interfaces to native code. The idea was to leverage a widely used tool
that could generate code for different languages.  SWIG is written in
\perl{} and was somewhat complex. It is more desirable to write the
code generation mechanism in \R{} itself rather than another language.
This is because it is more natural and also because more users can
understand the code and contribute and extend the development.  Also,
as the name suggests, the focus of SWIG is to generate bindings to
existing native code.  It does not allow us to examine the body of the
routines.


Many years ago, we adapted and embedded the little \C{} compiler
(lcc)~\cite{bib:lcc} in \S.  This gave us many of the features we
obtain with \libclang, allowing us to access information about
routines, data structures, etc. in source code.  It does not support
\Cpp. Unfortunately, it did not readily deal with extensions to the
\C{} language used in the Linux header files. As a result, other than
extending the code base, we couldn't continue to use it for our
purposes.
 
Microsoft Windows provides facilities for accessing the information
about routines and data structures in native code via its type
libraries.  I developed the
\Rpkg{SWinTypeLibs}~\cite{bib:SWinTypeLibs} package to read this
information generally. This gives us access to much of the information
we need. Of course, it is specific to Microsoft Windows. It works on
previously compiled code, not the source code.  Accordingly, it gives
us no access to the bodies of the native code.


% SWIG
% gcc-xml
% Slcc
% Type libraries on Windows.

All of the approaches are reasonable and have their
advantages. However, \libclang{} is a significant improvement over all
of these other approaches for various different reasons. It provides
most of the infrastructure we need and removes the burden for
conusmers to create this.  It is flexible but not excessively complex.
It is tied to serious, ongoing projects and so continues to evolve. It
allows us to work in the language in which we want to program,
i.e. \R.  It provides access to all aspects of the native code, not
just the declarations. It processes both \C{} and \Cpp.


\section[Concepts of libclang]{Concepts of \libclang}\label{sec:libclangConcepts}

Translation unit,  AST and cursors, traversing the tree, 

\section[The RCIndex package]{The \Rpkg{RCIndex} package}\label{sec:RCIndex}
The \Rpkg{RCIndex} provides numerous high-level functions that return
information about \C{} code.

\subsection{High-level Functionality}

% Talk about filtering based on the file name. (Implement this
% consistently).  Mention can do it afterwords or during the collection.


We might consider functions that take a source file and extract one or
more of the types of top-level elements in that file, e.g. routines,
data structures, enumeration definitions, \Cpp{} class definitions as
the very highest-level functions.  The \R{} user can call these
functions with just the name of the file and perhaps additional
arguments for the parser and the results are returned.  The user
doesn't have to write any code to manipulate or traverse the parse
tree (AST).  She doesn't have to necessarily create a translation unit
before calling one of these functions. As such, they are ``atomic''.
(There are occasssions, however, when it is benficial to create a
translation unit and traverse that several times to extract different
information on each traversal.)

The \Rfunc{getRoutines} does as its name suggests. It takes a file
name or an existing translation unit object and returns a list with an
element for each routine declaration or definition in the code.  Each
element contains the \Rclass{CXCursor} object representing the
routine, a list of the parameters giving their name and data types,
and the return type.  This is currently an S3 object of class
\Rclass{FunctionDecl}. Since this contains the cursor, we can query it
for the name of the routine, the file in which it is located, the
location in that file and so on. In other words, information that
we don't explicitly collect into the \R{} object can be determined
later when we use this description of the routine.

\Rfunc{getEnums} returns a list of the enumeration definitions in a
source file. Each element is an instance of the S4 class
\Rclass{EnumerationDefinition}.  Like the \Rclass{FunctionDecl}, this
contains a reference to the definition in case we want to query it
further at a later time.  The actual values in the enumeration are
stored in the \Rslot{values} slot as a named integer vector.  The
names are the symbolic names we should use, while the values are the
literal values to which these names correspond.  These values allow us
to cross the interface between \R{} and native code where there is no
existin association between the symbolic names.

\Rfunc{getDataStructures} returns a list of the data types defined in
a source file.


\Rfunc{getGlobalVariables} yields references to all of the non-local
variables within a source file.  From this, we have their names and
types.

\Rfunc{getCppClasses} traverses a translation unit, either pre-parsed
or a source file, and constructs a description for each of the \Cpp{}
classes it contains. Each class description is an instance of the
class \Rclass{C++Class}, with template classes using the derived class
\Rclass{TemplateC++Class}.  A class description contains the name of
the class, a list of its fields and another for its methods, and
references to the base/super class cursors.  The fields and methods
each have their type information and also the accessor qualifier,
i.e. \Cppkeyword{private}, \Cppkeyword{protected} or
\Cppkeyword{public}.


\subsection{Building Blocks}

\Rfunc{createTU}


\subsection{Writing visitor functions}
%Closures, reference classes
%Reference class and inheritance.


%XXX Correct this to talk about traversing any cursor, not just the TU.
The primary way to extract information from a translation unit is to
use the \Rfunc{visitTU} function.  This takes the name of a source
file or an already parsed translation unit.  The second argument is a
``visitor''.  This can take various forms but essentially it is an
\R{} function (or the address of a native symbol).  \libclang{}
iterates over some or all of the cursors in the translation unit and
calls the \R{} function for each each of these cursors, passing it
both the current cursor and its parent cursor, for context.  The
function can perform any computations it desires and typically
extracts any information it wants from each cursor. 
% Move this to after the closures and explain that we can always
% Recurse, but it can be a waste of time.
It then returns a
value to indicate to \libclang{} how it should continue to traverse
the cursor tree.  We can instruct \libclang{} to stop, to traverse
only the siblings of the current cursor, or to recursively process the
child cursors.  We can specify a different return value for each call
and so change how \libclang{} traverses the tree dynamically.  For
example, we might want to process all of the nodes in a particular
routine but no other routines. So we would return
\Rvar{CXChildVisit_Recurse} when we first encounter the \Rvar{CXCursor_FuncDecl}%??? 
with the name of interest.  When we encounter the start of an other routine, we return
\Rvar{CXCursor_Continue} or we can simply terminate the traversal with
\Rvar{CXCursor_Break}.

The visitor function typically stores information that persists across
calls to it by \libclang.  We then access these objects when
\libclang{} has completed traversing the tree.  While we may be
inclined to use global variables, e.g. in \R{}'s global environment or
interactive ``work space'', this is a bad idea.  Instead, we want to
use variables that are local to this particular visitor.  We can do
this in two basic ways -- reference classes or closures, also known as
lexical scoping.  In fact, these two are very similar.

We'll focus on a very simple example which merely
iterates over the sub-cursors and finds the name and location of each
routine in the translation unit.  We are looking for cursors at the
top-level of the translation unit which have a cursor kind
\Rvar{CXCursors_FunctionDecl}.  We ignore any other cursors.

We will store the name of the routine and also its line number.
We have to put these into a non-local variable within a call to our
visitor  function and then be able to retrieve them. This is a closure.
Creating a closure is quite simple but requires understanding how \R{}
finds variables.  One approach to creating a closure is to define a
function that returns one or more functions.  We often call this a
generator function.
We can define our generator function something like
\begin{RCode}
genRoutineLocations = 
function() 
{
   locations = integer()

    visitor = function(cur, parent) {
       if(cur$kind == CXCursor_FunctionDecl) 
         locations[[ getName(cur) ]] <<- getLocation(cur)$location["line"]

        CXChildVisit_Continue
    }

    list(update = visitor, locations = function() locations)
}
\end{RCode}
There are several things to note.  Firstly, we have defined the
\Rfunc{genRoutineLocations} function. That is not what we will use to
traverse the tree. Instead, we call it to get a new pair of functions
returned in a list.  We will pass the \Rel{visitor} element of that
list to \Rfunc{visitCursor}.  Secondly, note the use of \verb+<<-+
when we assign the line number to the element of our \Rvar{locations}
vector.  Thirdly, note that the \Rvar{locations} vector is defined in
the body of \Rvar{genRoutineLocations}, not within our visitor
function. The visitor function merely modifies it.  Fourthly, our
second function in the returned list (\Rel{locations}) is an anoymous
function also defined during a call to \Rfunc{genRoutineLocations} and
it returns the current value of the \Rvar{locations} variable.  This
allows us to retrieve the results after they were added in the
different calls to our visitor function.  Lexical scoping and closures
are described in various texts about \R and \cite{bib:LexicalScoping}
is the definitive reference.

We can now use our generator function and obtain the line numbers of
the different routines. We do so with
\begin{RCode}
f = system.file("exampleCode", "fib.c", package = "RCIndex")
col = genRoutineLocations()
visitCursor(f, col$visitor)
col$locations()
\end{RCode}
Our call to \Rfunc{genRoutineLocations} returns the list of two
functions.
We pass the \Rel{visitor} element  to \Rfunc{visitCursor} (along with
the name of the source code file).
Then we retrieve the updated contents with the second function \Rel{locations}.


As an aside, note that in this particular case, we don't need to
recursively descend through the cursors as the routines will all be
immediate children of the translation unit.  Therefore, our visitor
function returns \Rvar{CXChildVisit_Continue} to move to the next
sibling and not down across its children.  This saves times processing
irrelevant cursors.

Closures are very powerful but confuse some \R{} programmers.
Reference classes may be more familiar, especially to those familiar
with classes in \Cpp, \Java{} or \Python.  The idea is that we define
a class that has methods that share variables.  Our methods will be
our \Rel{visitor} and \Rel{locations} functions in the generator
function above, and the fields or variables will be the
\Rvar{locations} numeric vector.  We can use the same code, but
aggregate it in a different way to define a reference class as:
\begin{RCode}
RoutineLocationVisitor =
setRefClass("RoutineLocationVisitor",
    fields = list(locations = "numeric"),
    methods = list(getLocations = function() 
                                     locations,
                   visitor = 
                       function(cur, parent) {
                         if(cur$kind == CXCursor_FunctionDecl) 
                            locations[[ getName(cur) ]] <<- 
                                   getLocation(cur)$location["line"]
                               
                         CXChildVisit_Continue
                       }))
\end{RCode}
Here we explicitly identify the shared variable and the methods.  The
only real difference is that we have call our function to retrieve the
\Rvar{locations} value \Rfunc{getLocations} rather than
\Rfunc{locations}.  This is merely to avoid have a field and a method
with the same name.

We can now use this reference class in almost exactly the same way we
used our  generator function:
\begin{RCode}
col = RoutineLocationVisitor()
visitCursor(f, col$visitor)
col$getLocations()
\end{RCode}
The \Rfunc{RoutineLocationVisitor} function returned by
\Rfunc{setRefClass} creates a new instance of this reference class.
We pass the \Rfunc{visitor} method to \Rfunc{visitCursor} and then
call the function \Rfunc{getLocations} to obtain the results.



You can use any reference class you want and then pass the visitor or
update method to \Rfunc{visitTU}.  However, it can be convenient to
define your reference class to extend the \Rclass{RefVisitor} class in
the \Rpkg{RCIndex} package. This allows you to pass the entire
reference object to \Rfunc{visitTU} and to get the results back
directly. This is merely syntactic sugar to simplify the
programming. It changes the code above to create the reference class,
call \Rfunc{visitCursor} and retrieve the results to the more succinct
\begin{RCode}
visitCursor(cur, RoutineLocationVisitor)
\end{RCode}
Essentially, extending the \Rclass{RefVisitor} class allows
\Rfunc{visitCursor} to identify the visitor and the result functions.
The only change we need to make when defining our reference class
is to add the \Rarg{contains} argument in the call to
\Rfunc{setRefClass}, i.e.
\begin{RCode}
setRefClass("RoutineLocationVisitor",
            contains = "RefVisitor",
            fields = list(locations = "numeric"),
            methods = list(getLocations = function() locations,
                       .....))
\end{RCode}



Similar to the reference class, we can use an S4 class
\Rclass{S4Visitor} to combine the visitor and result function
and pass the two together to \Rfunc{visitCursor}.
We typically create our two functions as before
via a generator function, and then combine the two 
functions into a formal object with, for example,
\begin{RCode}
col = genCollector()
visitCursor( cur, new("S4Visitor",  update = col$update, result = col$vars))
\end{RCode}
Again, the purpose of this \Rclass{S4Visitor} class is merely to 
allow \Rfunc{visitCursor} to identify the two functions.





\section{Examples}
In this section, we explore some more advanced examples of how to use
the \Rpkg{RCIndex} package to get different information from a
translation unit.  One of the best sources of such examples is the set
of high-level functions in the package itself,
e.g. \Rfunc{getRoutines}, \Rfunc{getCppClasses}.  We have used the
generator function and lexical scoping approach to implement a set of
collector functions that gather the different types of information we
want.

In section~\ref{sec:Introduction}, we provided numerous motivating
applications of being able to process native code.  We will present
partial/heuristic approaches to some of these.  One aim of these
examples is to illustrate how to traverse the translation unit and
sub-cursors and work on the structure of the information.  Another aim
is to illustrate how to work with \libclang's type system.

\subsection[Checking for garbage collection errors in native code for R]{Checking for garbage collection errors in native code for \R{}}
In this example, we will write code that can examine a \C{} routine
written to be called via \R's \Rfunc{.Call} interface and try to
identify if there are possible errors related to ensuring \R{} objects
are not garbage collected prematurely.  The \R{} API uses the macros
\Cfunc{PROTECT} and \Cfunc{UNPROTECT} to mark an object as being in
use and stop if from being garbage collected and to unmark a number of
protected objects.  For example, the \C{} code in
figure~\ref{fig:ProtectCorrect} creates two new \R{} objects and
protects them both, performs some computations that populate the
objects and then unprotects both with a call \Cexpr{UNPROTECT(2)}.  In
contrast, the code in figure~\ref{fig:ProtectIncorrect} does not
protect the \R{} object it creates.  It is quite possible that after
allocating \Cvar{ans}, \R{} will release that object when it allocates
\Cvar{names} in the next expression. At that point, the memory is
corrupted and errors and crashes are likely.
In other cases, we might protect the \R{} objects we create,
but fail to unprotect them, or at least some of them.

\noindent
\begin{figure}[ht]
\begin{minipage}[t]{0.45\linewidth}
\centering
\begin{CCode}
SEXP
R_foo_correct(SEXP r_x)
{
  SEXP ans, names;
  int n = Rf_length(r_x);
  double *x = REAL(x);
  PROTECT(ans = NEW_NUMERIC(n));
  PROTECT(names = NEW_CHARACTER(n));
  for(int i = 0; i < n; i++) {
    char *str;
    REAL(ans)[i] = foo(x[i], &str);
    SET_STRING_ELT(names, i, 
                     mkChar(str));
  }
  SET_NAMES(ans, names);
  UNPROTECT(2);
  return(ans);
}
\end{CCode}
\caption{This is a simple and correct \C{} routine
that protects and unprotects the \R{} objects it creates.}
\label{fig:ProtectCorrect}
\end{minipage}
\hspace{0.5cm}
\begin{minipage}[t]{0.45\linewidth}
\centering
\begin{CCode}
SEXP
R_foo_incorrect(SEXP r_x)
{
  SEXP ans, names;
  int n = Rf_length(r_x);
  double *x = REAL(x);
  ans = NEW_NUMERIC(n);
  names = NEW_CHARACTER(n);
  for(int i = 0; i < n; i++) {
    char *str;
    REAL(ans)[i] = foo(x[i], &str);
    SET_STRING_ELT(names, i, 
                       mkChar(str));
  }
  SET_NAMES(ans, names);
  return(ans);
}

\end{CCode}
\caption{This version does not protect the \R{} objects and, hence,
  also  doesn't unprotect them.}
\label{fig:ProtectIncorrect}
\end{minipage}
\end{figure}


%XXX Clean this up wrt. to macro expansions. Ignore them  but just
%mention it is possible.
We will develop an \R{} function that will attempt to check for common
problems related to garbage collection.  Our function take the
\libclang{} cursor for a routine and will then traverse the
nodes/cursors throughout that tree. It will find calls to known
routines that alloc \R{} objects (specifically \Cfunc{Rf_allocVector})
and also calls to \Cfunc{Rf_protect} and \Cfunc{Rf_unprotect}.  At its
very simplest, we can count the number of allocations, the number of
calls to \Cfunc{Rf_protect} and attempt to determine the value passed
to \Cfunc{Rf_unprotect}.  If these don't match, we can indicate a
potential error.  This is a simple heuristic approach, but it may
catch some cases.

What about \Cfunc{NEW_NUMERIC}, \Cfunc{PROTECT} and \Cfunc{UNPROTECT}?
In some cases, these are acutally pre-processor macros and in others
they are actually routines. They expand to
\Cexpr{Rf_allocVector(REALSXP, n)} and calls to \Cfunc{Rf_protect} and
\Cfunc{Rf_unprotect}.  We can find these expansions and substitutions
in the translation unit. Alternatively, we can use our knowledge of
how the \R{} API works.


Since we will traverse over the sub-cursors, we need a visitor
function. We'll use a closure, but again, we could use a reference
class.  We need variables in which we can record the number of calls
to each of \Cfunc{Rf_allocVector}, \Cfunc{R_protect} and
\Cfunc{R_unprotect}.  We'll call these variables \Rvar{numAllocs},
\Rvar{numProtectCalls} and \Rvar{numUnprotectCalls}, respectively.
For the first two routines, we just increment the current value
in the corresponding \R{} variable.
For \Cfunc{Rf_unprotect}, we'll collect the argument for each call to
\Cfunc{Rf_unprotect}. This might be a literal value, e.g. $2$ in our
example, or a variable or a general expression.

We have now identified the basic strategy for our visitor function.
Next, we need to specify the details for the different types of
cursors we encounter. A call to any of our routines will be identified
by a cursor with a kind \Rvar{CXCursor_CallExpr}.  We can get the name
of the routine being called using \Rfunc{getName} applied to this
cursor. (There are other ways also, but this is the simplest.)  Based
on this name, we update the relevant counter or ignore the call.  If
the call is to \Cfunc{Rf_unprotect}, we have to arrange to get its
argument.  We can use either of two approaches for this.  We can set a
flag in our closure to indicate we are processing the sub-cursors of a
\Cfunc{Rf_unprotect} call and then subsequent calls to our visitor
function can check this and interpret the sub-cursors
appropriately. Alternatively, we can explicitly traverse the children
of the current call cursor to determine the argument.
We'll use the former approach.

We define our generator  function as
\begin{RCode}
genProtectAnalyzer =
function()
{
  numAllocs = 0  
  numProtectCalls = 0
  numUnprotectCalls = numeric()

  inUnprotect = FALSE  
  allocCounterVarName = ""    
  unProtectParent = NULL

  update = function(cur, parent)    {
    k = cur$kind

    if(inUnprotect && identical(unProtectParent, cur) ) {
       unProtectParent <<- NULL
       inUnprotect <<- FALSE
     }
    
   if(k == CXCursor_CallExpr) {
      fn = getName(cur)
      print(fn)
      if(fn == "PROTECT" || fn == "Rf_protect")
        numProtectCalls <<- numProtectCalls + 1
      else if(fn == "UNPROTECT" || fn == "Rf_unprotect") {
        numUnprotectCalls <<- numUnprotectCalls
        unProtectParent <<- parent
        inUnprotect <<- TRUE
      } else if(fn %in% c("Rf_allocVector", "NEW_NUMERIC", "NEW_INTEGER", "NEW_LOGICAL", "NEW_CHARACTER"))
          numAllocs <<- numAllocs + 1
    } else if(inUnprotect) {
       if(k == CXCursor_IntegerLiteral) {
          val = getCursorTokens(cur)["Literal"]
          numUnprotectCalls <<-  c(numUnprotectCalls, as.integer(val))
       } else if(k == CXCursor_FirstExpr) {
           allocCounterVarName <<- getName(cur)
       }
    }
    
    CXChildVisit_Recurse
  }

  list(update = update,
       info = function() {
                 list(numProtectCalls = numProtectCalls,
                      numUnprotectCalls = numUnprotectCalls,
                      inUnprotect = inUnprotect,
                      numAllocs = numAllocs)})
\end{RCode}
%$
For the present, ignore that second expression
in our visitor function, the \Rfunc{if} expression.
We'll discuss this below.



Before we can check the routine, we have to read it into \R{}.
We can do this with \Rfunc{getRoutines} via
\begin{RCode}
f = system.file("exampleCode", "protectIncorrect.c", package = "RCIndex")
r = getRoutines(f, f, includes = sprintf("%s/include", R.home()))
\end{RCode}
Note that we specified the include directories so that
the parser could find \file{Rinternals.h} and the other 
\R{} header files.

We create our visitor function by calling the generator function
and then we pass the visitor to \Rfunc{visitCursor},
along with our routine we want to check:
\begin{RCode}
v = genProtectAnalyzer()
visitCursor(r$R_foo_incorrect, v$update)
v$info()
\end{RCode}
The output is 
\begin{ROutput}
$numProtectCalls
[1] 0

$numUnprotectCalls
numeric(0)

$numAllocs
[1] 2
\end{ROutput}
We can see there are two allocations and no calls to protect or
unprotect.  
In constrast, when we run this on the correct version, 
each of the three counts has a value of $2$.

\begin{comment}
f = system.file("exampleCode", "protectCorrect.c", package = "RCIndex")
r = getRoutines(f, f, includes = sprintf("%s/include", R.home()))
v = genProtectAnalyzer()
visitCursor(r$R_foo, v$update)
v$info()
\end{comment}
%$

We should revisit the \Rfunc{if} expression near the beginning
of our visitor function:
\begin{RCode}
if(inUnprotect && identical(unProtectParent, cur) ) {
   unProtectParent <<- NULL
   inUnprotect <<- FALSE
}
\end{RCode}
The purpose of this is to ensure that we don't continue to collect
information from other cursors after we have processed any call to
\Cfunc{UNPROTECT}.  If we didn't have this, the variable
\Rvar{inUnprotect} would remain \Rtrue.  As a result, if there are
\C{} expressions in the body of the routine after the call to
\Cfunc{UNPROTECT} and they contain literal values or
\Rvar{CXCursor_FirstExpr} cursors, we will continue to accumulate
these as if they related to the \Cfunc{UNPROTECT} call.  As a result,
we need to determine when we have finished examining the sub-cursors
of the \Cfunc{UNPROTECT} call.  To do this, we record the parent
cursor of the \Cfunc{UNPROTECT} call in \Rvar{unProtectParent}. Each
time our visitor is called, we can compare the current cursor to this
cursor and if they are the same, we are back to a sibling of the
\Cfunc{UNPROTECT} call.

Setting and unsetting a state variable across calls to a visitor
function can be complex and often requires clear thinking.
We could have used the other approach which was to detect
the call to \Cfunc{UNPROTECT} and immediately determine its arguments
using \Rfunc{children}. For example, we could have added the code
\begin{RCode}
arg = children(cur)[[2]]
if(arg$kind == CXCursor_IntegerLiteral)
   as.integer(getCursorTokens(arg)["Literal"])
\end{RCode}
%$
This makes the visitor function a great deal simpler and also slightly
faster as we can avoid traversing the sub-cursors again.  However,
sometimes we have to use state, in particular when the information we
need to extract is not in sub-cursors but in silbing cursors and their
descendants.


As a final remark about this example, we could make this more general
and sophisticated.  A reasonably common idiom is to use a variable to
count the number of calls to \Cfunc{PROTECT} we make, incrementing it
each time. Then we pass this to \Cfunc{UNPROTECT}, e.g.
\begin{CCode}
 int n = 0;
 PROTECT(a = Rf_allocVector(...)); n++;
 PROTECT(b = Rf_allocVector(...)); n++;
 PROTECT(c = Rf_allocVector(...)); n++;
 PROTECT(d = Rf_allocVector(...)); n++;

 UNPROTECT(n);
\end{CCode}
 It is easy to omit incrementing the counter in one
or more cases. We can try to trace this by identifying the variable in
the call to \Cfunc{UNPROTECT} and then finding out where it is
incremented and try to find cases where it is not.
% Show example code







\section{Future Work}

Given the basic bindings, we plan to develop additional functionality
to process different aspects of native code. We are interested in
processing the bodies of the routines to understand them better.

The functions to generate bindings might benefit from being
reorganized and made more general and extensible.

We have created bindings for most of the facilities provided by
\libclang.  However, we have ignored the code-completion routines and
the various indexing facilities.  Now that we have the basic tools
provided by \Rpkg{RCIndex} in \R, we can programmatically generate
bindings to these routines. 


\bibliography{rclang}


\end{document}
