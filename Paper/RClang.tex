\documentclass[article]{jss}

%%%%%%%%%%%%%%%%%%%%%%%%%%%%%%
%% declarations for jss.cls %%%%%%%%%%%%%%%%%%%%%%%%%%%%%%%%%%%%%%%%%%
%%%%%%%%%%%%%%%%%%%%%%%%%%%%%%

%% almost as usual
\author{Duncan Temple Lang\\University of California at Davis}
\title{RCIndex: Accessing C/C++ Code in R with libclang}

%% for pretty printing and a nice hypersummary also set:
\Plainauthor{Duncan Temple Lang}
\Plaintitle{RCIndex: Accessing C/C++ Code in R with libclang}
\Shorttitle{RCIndex}

%% an abstract and keywords
\Abstract{ 
  The ability to get information about \C/\Cpp{} code routines
  and data structures can allow us to do many things in an intrepreted
  language such as \R.  We use \libclang{}, a flexible, embeddable
  library, to develop \R{} functionality to obtain information about
  native code.  We describe the \Rpkg{RCIndex} package which provides
  high-level functionality to access this information and also
  lower-level approaches to query and manipulate other aspects of
  native code.  We describe how to use the package and scenarios in
  which it is useful.
} 
\Keywords{\R, \Rpkg{RCIndex} package}
\Plainkeywords{R, RCIndex} %% without formatting
%% at least one keyword must be supplied

%% publication information
%% NOTE: Typically, this can be left commented and will be filled out by the technical editor
%% \Volume{13}
%% \Issue{9}
%% \Month{September}
%% \Year{2004}
%% \Submitdate{2004-09-29}
%% \Acceptdate{2004-09-29}

%% The address of (at least) one author should be given
%% in the following format:
\Address{
  Duncan Temple Lang\\
  4210 Math Sciences Building, \\
  University of California at Davis \\
  One Shields Avenue\\
  Davis, CA 95616\\
  E-mail: \email{duncan@r-project.org}\\
  URL: \url{http://www.omegahat.org}
}

\usepackage[T1]{fontenc}
\catcode`\_=12

\def\C{\proglang{C}}
\def\perl{\proglang{PERL}}
\def\Cpp{\proglang{C$++$}}
\def\R{\proglang{R}}
\def\llvm{\proglang{LLVM}}
\def\Rpkg#1{\pkg{#1}}
\def\Rfunc#1{\textsl{#1()}}
\def\Rop#1{\texttt{#1}}
\def\Rvar#1{\textsl{#1}}
\def\Cfunc#1{\textit{#1()}}
\def\Cvar#1{\textit{#1}}
\def\file#1{\textbf{#1}}
\def\Ctype#1{\texttt{#1}}
\def\Rclass#1{\textit{#1}}
\def\Rslot#1{\textbf{#1}}
\def\Rarg#1{\textbf{#1}}
\def\Roption#1{\dquote{\textsl{#1}}}
\def\Cppkeyword#1{\textbf{#1}}

\def\libclang{\textbf{libclang}}
\def\clang{\textbf{clang}}
\def\gcc{\textit{GCC}}
\def\XML{\textit{XML}}

\def\dquote#1{``#1''}

\def\ShFlag#1{\textit{#1}}

\DefineVerbatimEnvironment{RCode}{Verbatim}{fontshape=sl}
\DefineVerbatimEnvironment{CCode}{Verbatim}{fontshape=tt}
\DefineVerbatimEnvironment{ShCode}{Verbatim}{fontshape=it}
\DefineVerbatimEnvironment{ROutput}{Verbatim}{fontshape=bf}

\begin{document}

\section{Motivation}\label{sec:Introduction}

We typically think of \C{} and \Cpp{} code as something we write,
compile and call. It is rarely the input to anything but the compiler.
However, such code is a source of potentially useful information that
we can use in statistical and scientific computing.  In order to
leverage this information, we need to be able to access it in a
structured manner in a form that we can compute on and in the
programming environment in which we want to use it.
\libclang{}~\cite{bib:libclang,bib:libclangSlides} has
emerged as a powerful, industrial-strength library that provides
valuable functionality for working with \C/\Cpp{} source code and that
we can embed in \R (and many other languages).

There are several applications of being able to programmatically
understand \C/\Cpp{} code.

\textbf{Registering \R-callable routines}: When we write \C/\Cpp code
to use in an \R{} package, it is useful to explicitly register the
routines that we can call via any of the
\Rfunc{.C}/\Rfunc{.Call}/\Rfunc{.External} interfaces.  When the
routines are registered, \R{} can help to identify potential errors in
calling them.  \R{} can detect an incorrect number of arguments for a
routine, or that the types are incorrect, e.g. an integer vector when
a numeric vector is expected. Registration also allows us to resolve
the symbols just once rather than each time we call a routine, and it
also allows us to use different symbols to refer to the routines.  It
is convenient to be able to run \R{} code to identify the \R-callable
routines and to generate the registration information
programmatically.  As we change these routines, we can update the
registration information with little effort and ensure the information
is synchronized.


\textbf{Generating bindings to native libraries}: Numerous \R{}
packages provide interfaces to existing \C/\Cpp{} libraries.  This
typically involves manually creating an \R-callable wrapper routine
for each routine of interest in the third-party library and also a
corresponding \R{} function that invokes the wrapper routine, having
coerced the \R{} arguments to the appropriate form.  This is often
quite straightforward, but both time-consuming and error-prone.  This
makes for unnecessarily lengthy write-debug-test cycles.  Instead,
we'd like to be able to programmatically read the information about
the third-party routines and data types and generate the wrapper
routines and \R{} functions without human intervention, or at least
minimal, generic intervention.  If we could generate these
``bindings'' programmatically, the \R{} programmer can spend time
creating higher-level functionality using these primitives.

\textbf{Dynamic calls to native routines}: The
\Rpkg{rdyncall}~\cite{bib:rdyncall} and \Rpkg{Rffi}~\cite{bib:Rffi}
\R{} packages avoid having to explicitly create the wrapper routines
and \R{} functions to interface to existing \C{} routines. Instead,
they both provide a dynamic mechanism to call arbitrary native
routines.  However, both approaches require a description of the
target routines.  Again, we want to obtain this information
programmatically and then we can easily generate these descriptions
and remove humans from the process.

\textbf{Understanding third-party libraries interactively}: When we
interface to third-party libraries, we typically read documentation to
identify and understand the important routines and data structures.
In some situations, it can be convenient to find this information
interfactively within an environment such as \R.  Rather than reading
static document, we can query the code for information such as how
often a particular data type is returned by a routine or passed as an
argument? or what idioms does the library use?  We can use \R's
graphics capabilities to visualize the code and which routines call
which other routines.

\textbf{In-line documentation as comments}: Often third-party native
libraries contain document in comments adjacent to the corresponding
routines and data structures.  It is convenient to be able to easily
access this documentation and potentially reuse it as \R{}
documentation for functions that interface to the routines.

\textbf{Compiling \R{} code}:  Recently, we have been developing \R{} facilities for compiling \R{}
code to native instructions to by-pass the \R{} interpreter.  This
allows us to rewrite and translate \R{} code so that it is essentially
native code and can call other existing native routines, for example,
in the \R{} engine or standard \C{} libraries.  This results in
significant speedup.  However, to make this work, we need to know the
signature -- parameter types and return type -- of these native
routines.  Again, if we can find these programmatically, we greatly
simplify and improve the entire process of generating the code.

\textbf{Determining memory management and mutability of inputs and
  outputs}: When we call existing native routines, we often want need
to know whether we are in charge of the memory for the inputs or
whether the called routine is responsible.  Often, programmers omit
important information about whether a parameter is modified within a
routine or if it is constant.  This is important information as it
allows us to differentiate between an array of values passed as a
pointer, or a scalar whose contents are changed.  We would like to be
able to analyze the body of the routine to be able to determine if and
how it modifies its parameters so that we can avoid making copies of
data, if possible.


\textbf{Software as Data}: While we may not think of code as data,
analyzing software is an important field.  Software defines several
networks relatd to i) which routines call which other routines, ii)
the hierarchical structure of \Cpp{} classes, iii) which files
\texttt{\#include} other files, and so on.  We can find which
global/non-local variables are used in which routines to help identify
problems with parallelization and potential refactorzation of the
code.  We can combine this data with version control history to better
understand software projects.


\textbf{Detecting errors in native code for \R}: \R's package
mechanism provides a powerful set of tests and checks to potential
errors in \R{} code.  These are very useful and can identify errors
such as misspelled variable or function names before the code is run.
It would be valuable to perform similar tests on \C{} code in \R{}
packages.  We might identify common issues such as not protecting
allocated \R{} objects from garbage collection.  We could do this by
analyzing uses of \Cfunc{PROTECT} and \Cfunc{UNPROTECT} calls and
ensuring there is an appropriate correspondance.  These are
\R-specific checks and will not be done by other code-analyzing
software.


We can also find ``dead'' routines that are never called by other
routines and so help to reduce the code.

We hope these applications motivate the utility of being able to
navigate native code in \R and indicate that there are many more.  In
this paper, we describe how to use the \R{} interface to \libclang.
We start in section~\ref{sec:Whylibclang} by comparing \libclang{}
with other approaches we and others have pursued.  We then describe
the fundamental concepts of \libclang{} in
section~\ref{sec:libclangConcepts}.  We follow this in
section~\ref{sec:RCIndex} by discussing the \R{} interface

%documentation in comments - see RCUDA.

\section[Why libclang?]{Why \libclang?}\label{sec:Whylibclang}

I have explored how we can get information about \C/\Cpp{} code in
\R{} for many years and have used various different approaches.
\libclang{} appears to be the most robust, stable and flexible so far.
\clang{} and \libclang{} are very active projects and important pieces
of software. This means they will continue to improve, evolve and be
maintained.  The API (Application Programming Interface) is
intentionally stable.  As we will see later, \libclang{} does the hard
work in resolving types uniquely and simplifying the complexities of
native code.  \libclang{} allows us to readily associate concepts in
the parse tree with precise locations in the text of the source
code. \libclang{} doesn't allow us to modify the parse tree and
rewrite native code. However, there is a quite different and more
advanced API related to \libclang{} that does and some of the concepts
will carry over to that, should we need those facilities in \R.

In the past, I have used the GNU compiler (\gcc)~\cite{bib:GCC} to
export information about the source code using its
\ShFlag{-fdump-translation-unit} command-line argument.  This outputs
many of the details of the source code in a reasonably low-level
format. I developed the \Rpkg{RGCCTranslationUnit}~\cite{bib:RGCCTU}
package to both parse the output from \gcc{} and also to collate the
information into higher-level descriptions of routines, data
structures, and so on.  Unfortunately, the format of \gcc's output is
not well document, and the information it exports has changed over
releases.  Generally, this a valuable source of information, but not
as stable or as readily usable or developed as \libclang.

GCC-XML is an extension to \gcc{} that, like the
\ShFlag{-fdump-translation-unit} option, can export information about
routines and data structures contained in \C/\Cpp{} source code.  As
the name suggets, it exports in an \XML{} format.  We can then read
this into \R{} with, for example, the \Rpkg{XML}~\cite{bib:RSXML}
package.  We can convert the \XML{} content into descriptions of the
routines and types.  Again, we have to implement this type registry.
Also, GCC-XML does not give us access to the bodies of the routines
and so we cannot address several of the applications we discussed in
section~\ref{sec:Introduction}.


I also explored SWIG (Simplified Wrapper and Interface
Generator)~\cite{bib:SWIG} as a means to programmatically generate
interfaces to native code. The idea was to leverage a widely used tool
that could generate code for different languages.  SWIG is written in
\perl{} and was somewhat complex. It is more desirable to write the
code generation mechanism in \R{} itself rather than another language.
This is because it is more natural and also because more users can
understand the code and contribute and extend the development.  Also,
as the name suggests, the focus of SWIG is to generate bindings to
existing native code.  It does not allow us to examine the body of the
routines.


Many years ago, we adapted and embedded the little \C{} compiler
(lcc)~\cite{bib:lcc} in \S.  This gave us many of the features we
obtain with \libclang, allowing us to access information about
routines, data structures, etc. in source code.  It does not support
\Cpp. Unfortunately, it did not readily deal with extensions to the
\C{} language used in the Linux header files. As a result, other than
extending the code base, we couldn't continue to use it for our
purposes.
 
Microsoft Windows provides facilities for accessing the information
about routines and data structures in native code via its type
libraries.  I developed the
\Rpkg{SWinTypeLibs}~\cite{bib:SWinTypeLibs} package to read this
information generally. This gives us access to much of the information
we need. Of course, it is specific to Microsoft Windows. It works on
previously compiled code, not the source code.  Accordingly, it gives
us no access to the bodies of the native code.


% SWIG
% gcc-xml
% Slcc
% Type libraries on Windows.

All of the approaches are reasonable and have their
advantages. However, \libclang{} is a significant improvement over all
of these other approaches for various different reasons. It provides
most of the infrastructure we need and removes the burden for
conusmers to create this.  It is flexible but not excessively complex.
It is tied to serious, ongoing projects and so continues to evolve. It
allows us to work in the language in which we want to program,
i.e. \R.  It provides access to all aspects of the native code, not
just the declarations. It processes both \C{} and \Cpp.


\section[Concepts of libclang]{Concepts of \libclang}\label{sec:libclangConcepts}

Translation unit,  AST and cursors, traversing the tree, 

\section[The RCIndex package]{The \Rpkg{RCIndex} package}\label{sec:RCIndex}
The \Rpkg{RCIndex} provides numerous high-level functions that return
information about \C{} code.

\subsection{High-level Functionality}

% Talk about filtering based on the file name. (Implement this
% consistently).  Mention can do it afterwords or during the collection.


We might consider functions that take a source file and extract one or
more of the types of top-level elements in that file, e.g. routines,
data structures, enumeration definitions, \Cpp{} class definitions as
the very highest-level functions.  The \R{} user can call these
functions with just the name of the file and perhaps additional
arguments for the parser and the results are returned.  The user
doesn't have to write any code to manipulate or traverse the parse
tree (AST).  She doesn't have to necessarily create a translation unit
before calling one of these functions. As such, they are ``atomic''.
(There are occasssions, however, when it is benficial to create a
translation unit and traverse that several times to extract different
information on each traversal.)

The \Rfunc{getRoutines} does as its name suggests. It takes a file
name or an existing translation unit object and returns a list with an
element for each routine declaration or definition in the code.  Each
element contains the \Rclass{CXCursor} object representing the
routine, a list of the parameters giving their name and data types,
and the return type.  This is currently an S3 object of class
\Rclass{FunctionDecl}. Since this contains the cursor, we can query it
for the name of the routine, the file in which it is located, the
location in that file and so on. In other words, information that
we don't explicitly collect into the \R{} object can be determined
later when we use this description of the routine.

\Rfunc{getEnums} returns a list of the enumeration definitions in a
source file. Each element is an instance of the S4 class
\Rclass{EnumerationDefinition}.  Like the \Rclass{FunctionDecl}, this
contains a reference to the definition in case we want to query it
further at a later time.  The actual values in the enumeration are
stored in the \Rslot{values} slot as a named integer vector.  The
names are the symbolic names we should use, while the values are the
literal values to which these names correspond.  These values allow us
to cross the interface between \R{} and native code where there is no
existin association between the symbolic names.

\Rfunc{getDataStructures} returns a list of the data types defined in
a source file.


\Rfunc{getGlobalVariables} yields references to all of the non-local
variables within a source file.  From this, we have their names and
types.

\Rfunc{getCppClasses} traverses a translation unit, either pre-parsed
or a source file, and constructs a description for each of the \Cpp{}
classes it contains. Each class description is an instance of the
class \Rclass{C++Class}, with template classes using the derived class
\Rclass{TemplateC++Class}.  A class description contains the name of
the class, a list of its fields and another for its methods, and
references to the base/super class cursors.  The fields and methods
each have their type information and also the accessor qualifier,
i.e. \Cppkeyword{private}, \Cppkeyword{protected} or
\Cppkeyword{public}.


\subsection{Building Blocks}

\Rfunc{createTU}


\subsection{Writing visitor functions}

\section{Examples}

When faced with attemp


\section{Future Work}

Given the basic bindings, we plan to develop additional functionality
to process different aspects of native code. We are interested in
processing the bodies of the routines to understand them better.

The functions to generate bindings might benefit from being
reorganized and made more general and extensible.

We have created bindings for most of the facilities provided by
\libclang.  However, we have ignored the code-completion routines and
the various indexing facilities.  Now that we have the basic tools
provided by \Rpkg{RCIndex} in \R, we can programmatically generate
bindings to these routines. 


\bibliography{rclang}


\end{document}
